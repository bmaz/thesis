\chapter{Conclusion}

To conclude this thesis manuscript, we first summarize the work that we have presented, then we synthesize the main research perspectives opened by this work.

\section{Summary of the thesis}


Social media is a complex phenomenon that has both positive and negative effects on people's welfare \citep{Allcottetal2020}. This also holds true for its impact on traditional media. According to \citet{Cision2019}'s Global State of the Media Report, the bypassing of traditional media by social media is considered to be the biggest challenge facing journalism for the future. At the same time, journalists increasingly rely on social media to stay connected to sources and real-time news.

We built a new dataset encompassing nearly 70\% of all the tweets in French over a long time period and the content produced by the general information media outlets during the same time period. We develop new algorithms that allow us to study the propagation of news stories between social and mainstream media. Focusing on the stories that emerge first on Twitter, we show that their popularity on Twitter affects the coverage that the mainstream media devotes to these stories.

signal to draw inferences about consumers' preferences. In future research, we will investigate whether this happens at the expense of the quality of the information produced.

\section{Directions for future research}