Social media is a complex phenomenon that has both positive and negative effects on the welfare of people \citep{Allcottetal2020}. This also holds true for its impact on traditional media. According to \citet{Cision2019}'s Global State of the Media Report, the bypassing of traditional media by social media is considered as the biggest challenges facing journalism for the future. At the same time, journalists increasingly rely on social media to stay connected to sources and real-time news.

In this paper, we focus on an important dimension that has been overlooked in the discussions on the implications of the changes brought by social media: namely how it affects publishers' production and editorial decisions. To do so, we built a new dataset encompassing nearly 70\% of all the tweets in French over a long time period and the content produced by the general information media outlets during the same time period. We develop new algorithms that allow us to study the propagation of news stories between social and mainstream media. Focusing on the stories that emerge first on Twitter, we show that their popularity on Twitter affects the coverage mainstream media devote to these stories.

These findings shed a new light on our understanding of how editors decide on the coverage for stories, and have to be taken into account when discussing policy implications of the recent changes in media technologies. In particular, while social media compete with mainstream media for audience attention, they can also be used as a signal to draw inferences about consumers' preferences. In future research, we will investigate whether this happens at the expense of the quality of the information produced.