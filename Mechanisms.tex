In the previous section, we have documented a positive relationship between the popularity of an event on Twitter and the coverage it receives on mainstream media. In this section, we discuss the different mechanisms at play behind this relationship.
% , and show that this relationship can be interpreted as causal

%%%%%%%%%%%%%%%%%%%%%%%%%%%%%%%%%
\subsection{Journalists monitor Twitter}
%%%%%%%%%%%%%%%%%%%%%%%%%%%%%%%%%

First of all, the fact that a number of stories appear first on Twitter and that we observe high reactivity of the mainstream media might be due to the fact that journalists are closely monitoring Twitter. A growing literature in journalism studies indeed highlights that social media play an important role as a news source. For example, \citet{vonNordheimBoczekKoppers2018} examine the use of Facebook and Twitter as journalistic sources in newspapers of three countries; they show that Twitter is more commonly used as a news source than Facebook.\footnote{For additional evidence of Twitter as a reporting tool, see e.g. \citet{Vis2012}.} Further, \citet{McGregorMolyneux2018}, who have conducted an online survey experiment on working U.S. journalists, show that journalists using Twitter as part of their daily work consider tweets as newsworthy as headlines from the Associated Press.

The use of Twitter crosses many dimensions of sourcing, information-gathering, and production of stories \citep{Wihbeyetal2018}. Most media organizations actively encourage journalistic activity on social media.   Among the $4,222,734$ Twitter accounts for which we have data, $0.12\%$ are the accounts of journalists (see Table \ref{Tab:table_summary_users_all_first} above); while this might seem low, it is actually rather high compared to the share of the total adult population journalists represent.

To investigate the role played by monitoring, for all the media organizations included in our sample, we compute the list of their journalists present on Twitter, and investigate the heterogeneity of the effects depending on this variable. Table \ref{Tab:number_articles_negbinomial_cevent_heterogeneity_nb_journalist_accounts} reports the results. While the coefficient on the effect of the number of tweets on the media coverage is pretty much similar for the media that have a high number of journalists with a Twitter account (Columns (3) and (4)) than for those with only a few numbers (Columns (1) and (2)), the marginal effect is twice as high for the former. Hence, while it cannot entirely explain our findings, the monitoring of Twitter by journalists seems to play a role here.


%%%%%%%%%%%%%%%%%%%%%%%%%%%%%%%%%%%%%%%%%%%%%%%%%%%%%%%%%%%%%%%%%%%%%%
\begin{table}
\caption{Naive estimates: Media-level approach, Depending on the number of journalists with a Twitter account}
\begin{center}
	{
\def\sym#1{\ifmmode^{#1}\else\(^{#1}\)\fi}
\begin{tabular}{l*{4}{c}}
\hline\hline
                    &\multicolumn{2}{c}{Low no. of journalists with Twitter}&\multicolumn{2}{c}{High no. of journalists with Twitter}\\\cmidrule(lr){2-3}\cmidrule(lr){4-5}
                    &\multicolumn{1}{c}{(1)}         &\multicolumn{1}{c}{(2)}         &\multicolumn{1}{c}{(3)}         &\multicolumn{1}{c}{(4)}         \\
\hline
Number of articles  &                     &                     &                     &                     \\
Number of tweets    &       0.044\sym{***}&       0.038\sym{***}&       0.046\sym{***}&       0.039\sym{***}\\
                    &     (0.010)         &     (0.009)         &     (0.010)         &     (0.008)         \\
Seed's number of followers&       0.000         &      -0.000         &       0.000         &      -0.000         \\
                    &     (0.000)         &     (0.000)         &     (0.000)         &     (0.000)         \\
\hline
Media FEs           &  \checkmark         &  \checkmark         &  \checkmark         &  \checkmark         \\
Month \& DoW FEs    &  \checkmark         &  \checkmark         &  \checkmark         &  \checkmark         \\
Drop media          &                     &  \checkmark         &                     &  \checkmark         \\
Drop multiple       &                     &  \checkmark         &                     &  \checkmark         \\
Observations        &     127,770         &      94,890         &     383,310         &     284,670         \\
Clusters (events)   &       4,259         &       3,163         &       4,259         &       3,163         \\
Marginal Effect     &       0.010         &  \emph{0.008}       &       0.019         &  \emph{0.015}       \\
\hline\hline
\end{tabular}
}

\end{center}
\begin{spacing}{0.5}
	{\fns \textbf{Notes:} * p$<$0.10, ** p$<$0.05, *** p$<$0.01. The time period is July 2018 - September 2018. Models are estimated using a negative binomial estimation. Standard errors are clustered at the event level. An observation is a media-news event.  Columns (1) and (3) report the estimates for all the events that appear first on Twitter; in Columns (2) and (4)  we drop the events whose seed is the Twitter account of a media or of journalist (``media") as well as the events whose seed broke more than one event during our time period (``multiple"). All specifications include the seed's number of followers as a control, and day-of-the-week, month, and media fixed effects. In Columns (1) and (2) (respectively (3) and (4), we consider the media with a relatively low (respectively relatively high) number of journalists with a Twitter account. The number of tweets is computed \textit{before} the first news article in the event and is in thousand. More details are provided in the text.} 
\end{spacing}
\label{Tab:number_articles_negbinomial_cevent_heterogeneity_nb_journalist_accounts}
\end{table} 
%%%%%%%%%%%%%%%%%%%%%%%%%%%%%%%%%%%%%%%%%%%%%%%%%%%%%%%%%%%%%%%%%%%%%%

%However, this explains the relationship we observe on the extensive margin, but not on the intensive margin.


%\begin{center}
%\textbf{[TO BE COMPLETED]}
%\end{center}
%
%
%To see role played by social media, we also investigate media outlets' strategy on Twitter: in particular, how many times they tweet in the event. \textbf{TO BE DONE}



%%%%%%%%%%%%%%%%%%%%%%%%%%%%%%%%%
\subsection{Editorial decisions and popularity}
%%%%%%%%%%%%%%%%%%%%%%%%%%%%%%%%%
% Clicks bias or Signaling model?

The causal relationship between the popularity of a story on Twitter and its mainstream media coverage can be due do the existence of a clicks bias. This explanation is consistent with the results of \citet{SenYildirim2015} that show, using data from a leading English language Indian national daily newspaper, that editors' coverage decisions of online news stories are influenced by the observed popularity of the story, as measured by the number of clicks received.

In \citet{SenYildirim2015}'s framework (that builds on \citet{Latham2015}), the newspaper cares about the revenue generated by covering a story, which is assumed to be proportional to the number of readers. To test for this hypothesis here, we use the fact that our sample of media outlets include a lot of different media outlets, some of them relying on advertising revenues while others do not. Table \ref{Tab:number_articles_negbinomial_cevent_heterogeneity_advertising} presents our estimates depending on whether the media rely on advertising revenues (around 80\% of the media outlets in our sample do so). The order of magnitude of the estimated effects of popularity on Twitter on media coverage is more or less similar in both cases; if anything, the marginal effects are slightly higher for the media that do not rely on advertising online.


%%%%%%%%%%%%%%%%%%%%%%%%%%%%%%%%%%%%%%%%%%%%%%%%%%%%%%%%%%%%%%%%%%%%
\begin{table}
\caption{Naive estimates: Media-level approach, Depending on the reliance on advertising revenues}
\begin{center}
	{
\def\sym#1{\ifmmode^{#1}\else\(^{#1}\)\fi}
\begin{tabular}{l*{4}{c}}
\hline\hline
                    &\multicolumn{2}{c}{No advertising}         &\multicolumn{2}{c}{Advertising}            \\\cmidrule(lr){2-3}\cmidrule(lr){4-5}
                    &\multicolumn{1}{c}{(1)}         &\multicolumn{1}{c}{(2)}         &\multicolumn{1}{c}{(3)}         &\multicolumn{1}{c}{(4)}         \\
\hline
Number of articles  &                     &                     &                     &                     \\
Number of tweets    &       0.049\sym{***}&       0.044\sym{***}&       0.048\sym{***}&       0.041\sym{***}\\
                    &     (0.010)         &     (0.011)         &     (0.009)         &     (0.007)         \\
Seed's number of followers&       0.000\sym{*}  &      -0.000\sym{**} &       0.000         &      -0.000         \\
                    &     (0.000)         &     (0.000)         &     (0.000)         &     (0.000)         \\
\hline
Media FEs           &  \checkmark         &  \checkmark         &  \checkmark         &  \checkmark         \\
Month \& DoW FEs    &  \checkmark         &  \checkmark         &  \checkmark         &  \checkmark         \\
Drop media          &                     &  \checkmark         &                     &  \checkmark         \\
Drop multiple       &                     &  \checkmark         &                     &  \checkmark         \\
Observations        &     140,547         &     104,379         &     596,260         &     442,820         \\
Clusters (events)   &       4,259         &       3,163         &       4,259         &       3,163         \\
Marginal Effect     &       0.017         &       0.014         &       0.011         &       0.009         \\
\hline\hline
\end{tabular}
}

\end{center}
\begin{spacing}{0.5}
	{\fns \textbf{Notes:} * p$<$0.10, ** p$<$0.05, *** p$<$0.01. The time period is July 2018 - September 2018. Models are estimated using a negative binomial estimation. Standard errors are clustered at the event level. An observation is a media-news event.  Columns (1) and (3) report the estimates for all the events that appear first on Twitter; in Columns (2) and (4)  we drop the events whose seed is the Twitter account of a media or of journalist (``media") as well as the events whose seed broke more than one event during our time period (``multiple"). All specifications include the seed's number of followers as a control, and day-of-the-week, month, and media fixed effects. In Columns (1) and (2) (respectively (3) and (4), we consider the media without online advertising (respectively with advertising revenues). The number of tweets is computed \textit{before} the first news article in the event and is in thousand. More details are provided in the text.} 
\end{spacing}
\label{Tab:number_articles_negbinomial_cevent_heterogeneity_advertising}
\end{table} 
%%%%%%%%%%%%%%%%%%%%%%%%%%%%%%%%%%%%%%%%%%%%%%%%%%%%%%%%%%%%%%%%%%%%%


Hence our results do not seem to be driven by short-term considerations generated by advertising revenue-bearing clicks. However, even absent such consideration, publishers may be willing to cover the stories that resonate the most. Or, to put it another way, news editors may aim at producing news consumers are interested in. But news editors do not know consumer's preferences; hence they can use the popularity of an event on Twitter as a signal that allows them to draw inferences about consumer preferences.

%Does this clicks bias happen at the expense of quality? For a given story importance -- and given the limited attention of readers\footnote{While media outlets do not face the same space constraint online that they face offline given space online is technically infinite, they still face the limited attention of the readers that can be considered as an implicit space constraint.}  -- publishers may be willing to cover the stories that resonate the most. Or, to put it another way, even absent short-term considerations generated by advertising revenue-bearing clicks, news editors may aim at producing news consumers are interested in. But news editors do not know consumer's preferences; hence they can use the popularity of an event on Twitter as a signal that allows them to draw inferences about consumer preferences.
%
%%To which extent is the popularity of a news story on Twitter a good signal of the traffic it will bring to mainstream media? While the data at our disposal do not allow us to properly estimate the audience received by each news article online (unfortunately, we do not have article-level audience data), we can provide a tentative answer to this question by relying on two different data sources. First, as described in Section \ref{Sec:DataNews}, we collect daily-level information on the number of unique visitors and the number of page views of the media outlets in our sample. \textbf{REGRESSION with daily-level variation}
%
%Second, we use the fact that most of the media outlets in our sample have one (or many) Twitter accounts, and post on Twitter the articles they publish online. \textbf{COMPUTE FOR THE MEDIA OUTLETS} share of the news article we capture every day that are then posted on Twitter.
% by using the number of tweets then generated by the news articles once they are posted on Twitter. What do we find?
% intensity of the media presence on Twitter: number of different accounts for each of the media outlets
%
%
%%%%%%%%%%%%%%%%%%%%%%%%%%%%%%%%%%
%\subsection{Competition for attention}
%%%%%%%%%%%%%%%%%%%%%%%%%%%%%%%%%%
%
%Lot of literature, mostly theoretical, that look both at competition between traditional media and aggregators and traditional media and social media.
%
%
%Twitter can also be used by the media to produce content at a relatively low cost; e.g. through articles reproducing content published on Twitter (rather than relying on interviews).
%
%
%
%%%%%%%%%%%%%%%%%%%%%%%%%%%%%%%%%%
%\subsection{Events that emerge first on Twitter}
%%%%%%%%%%%%%%%%%%%%%%%%%%%%%%%%%%
%
%Finally, to improve our understanding of the mechanisms at play, we investigate what happens for the events that merge first on Twitter.
%Main issue with these events: how to identify causal effect? but still of interest to see what happens.
%
%Set of events: 
%Stat des...
%
%\textbf{A COMPLETER}