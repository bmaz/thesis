\chapter*{Acknowledgements}

First of all, I would like to solemnly thank the members of my jury who agreed to evaluate my work. In particular, I would like to thank Marianne Clausel and Dominique Cardon for agreeing to report my thesis in this very particular context of lockdown, where no one knew exactly what their schedule would be in the coming months. Thank you to you, as well as to Ekaterina Zhuravskaya, Fabien Tarissan and Julien Velcin for taking part in my thesis defence on August 31, a date on which everyone would undoubtedly have liked to enjoy the summer a little longer.


My thanks go then to my thesis director, Céline Hudelot, and my co-directors, Julia Cagé and Nicolas Hervé. Céline, thank you for agreeing to supervise this bi-disciplinary thesis, which probably gave you more work than a thesis entirely in your discipline. Thank you also for taking the time come to INA, or to meet me at Centrale or in Paris. Finally, thank you for your methodological advice, your scientific rigour and your careful rereading. Julia, thank you for your energy and kindness, which encouraged me during the last months of my thesis. Thank you for having welcomed me in your office so often, and for having sometimes left me the keys. Finally, thank you for your hard work to help me make sense of the data. Nicolas, thank you for having been my daily contact at INA, for having answered all my questions without ever getting impatient, and for having made the completion of my thesis your priority among all your responsibilities. You made sure that I always had the space I needed to store my tweets, and the computing power to index and analyse them, and you were always concerned about my working conditions. I would also like to add a word of thanks to Marie-Luce Viaud, who came up with the idea for this thesis and was my co-director until her death. I think we all still miss her luminous intelligence and humour, and I wish she could have been there to see this project through.


I was fortunate during my thesis to be able to count on exceptional colleagues. Thanks to Pierre Letessier for his technical advice, on Docker, Elasticsearch or nearest neighbor search, and to Zeynep Pehlivan who gave me the benefit of her long experience in collecting tweets. Thanks to Laurent Cabaret for his advice on Python optimization. I would also like to thank Agnès Saulnier, for all the times she drove me home from the INA, and never complained that I talked to her about my thesis. The same thanks goes to David Doukhan, who was leaving even later than Agnès. Thanks to Jean-Etienne Noiré, Marc Evrard and Thomas Petit, who helped to make my lunch breaks funnier. Finally, a thought for Haolin Ren, who shared my office for many months, and has now returned to China.


I also had the unconditional support of my family during these three years. I am extremely grateful to my parents, Martine Créac'h and Bernard Mazoyer, for always believing in my qualities, which gave me the strength to carry out this long work. Thanks to my father for his enormous work in correcting all my articles and this thesis manuscript, and to my brothers, Johan and Simon Mazoyer, who took the time to discuss with me and to share with me their own experience of the thesis. A huge thank you also to my parents-in-law, Martine Avenel-Audran and Maurice Audran, who welcomed me in their home throughout the confinement and made the final writing period much simpler. Finally, I would like to thank Martin for his patience and tenderness, and for all the happiness I owe him.